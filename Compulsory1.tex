% Options for packages loaded elsewhere
\PassOptionsToPackage{unicode}{hyperref}
\PassOptionsToPackage{hyphens}{url}
%
\documentclass[
]{article}
\title{Compulsory exercise 1: Group 37}
\usepackage{etoolbox}
\makeatletter
\providecommand{\subtitle}[1]{% add subtitle to \maketitle
  \apptocmd{\@title}{\par {\large #1 \par}}{}{}
}
\makeatother
\subtitle{TMA4268 Statistical Learning V2022}
\author{Oskar Jørgensen, Halvor Linder Henriksen}
\date{16 February, 2022}

\usepackage{amsmath,amssymb}
\usepackage{lmodern}
\usepackage{iftex}
\ifPDFTeX
  \usepackage[T1]{fontenc}
  \usepackage[utf8]{inputenc}
  \usepackage{textcomp} % provide euro and other symbols
\else % if luatex or xetex
  \usepackage{unicode-math}
  \defaultfontfeatures{Scale=MatchLowercase}
  \defaultfontfeatures[\rmfamily]{Ligatures=TeX,Scale=1}
\fi
% Use upquote if available, for straight quotes in verbatim environments
\IfFileExists{upquote.sty}{\usepackage{upquote}}{}
\IfFileExists{microtype.sty}{% use microtype if available
  \usepackage[]{microtype}
  \UseMicrotypeSet[protrusion]{basicmath} % disable protrusion for tt fonts
}{}
\makeatletter
\@ifundefined{KOMAClassName}{% if non-KOMA class
  \IfFileExists{parskip.sty}{%
    \usepackage{parskip}
  }{% else
    \setlength{\parindent}{0pt}
    \setlength{\parskip}{6pt plus 2pt minus 1pt}}
}{% if KOMA class
  \KOMAoptions{parskip=half}}
\makeatother
\usepackage{xcolor}
\IfFileExists{xurl.sty}{\usepackage{xurl}}{} % add URL line breaks if available
\IfFileExists{bookmark.sty}{\usepackage{bookmark}}{\usepackage{hyperref}}
\hypersetup{
  pdftitle={Compulsory exercise 1: Group 37},
  pdfauthor={Oskar Jørgensen, Halvor Linder Henriksen},
  hidelinks,
  pdfcreator={LaTeX via pandoc}}
\urlstyle{same} % disable monospaced font for URLs
\usepackage[margin=1in]{geometry}
\usepackage{color}
\usepackage{fancyvrb}
\newcommand{\VerbBar}{|}
\newcommand{\VERB}{\Verb[commandchars=\\\{\}]}
\DefineVerbatimEnvironment{Highlighting}{Verbatim}{commandchars=\\\{\}}
% Add ',fontsize=\small' for more characters per line
\usepackage{framed}
\definecolor{shadecolor}{RGB}{248,248,248}
\newenvironment{Shaded}{\begin{snugshade}}{\end{snugshade}}
\newcommand{\AlertTok}[1]{\textcolor[rgb]{0.94,0.16,0.16}{#1}}
\newcommand{\AnnotationTok}[1]{\textcolor[rgb]{0.56,0.35,0.01}{\textbf{\textit{#1}}}}
\newcommand{\AttributeTok}[1]{\textcolor[rgb]{0.77,0.63,0.00}{#1}}
\newcommand{\BaseNTok}[1]{\textcolor[rgb]{0.00,0.00,0.81}{#1}}
\newcommand{\BuiltInTok}[1]{#1}
\newcommand{\CharTok}[1]{\textcolor[rgb]{0.31,0.60,0.02}{#1}}
\newcommand{\CommentTok}[1]{\textcolor[rgb]{0.56,0.35,0.01}{\textit{#1}}}
\newcommand{\CommentVarTok}[1]{\textcolor[rgb]{0.56,0.35,0.01}{\textbf{\textit{#1}}}}
\newcommand{\ConstantTok}[1]{\textcolor[rgb]{0.00,0.00,0.00}{#1}}
\newcommand{\ControlFlowTok}[1]{\textcolor[rgb]{0.13,0.29,0.53}{\textbf{#1}}}
\newcommand{\DataTypeTok}[1]{\textcolor[rgb]{0.13,0.29,0.53}{#1}}
\newcommand{\DecValTok}[1]{\textcolor[rgb]{0.00,0.00,0.81}{#1}}
\newcommand{\DocumentationTok}[1]{\textcolor[rgb]{0.56,0.35,0.01}{\textbf{\textit{#1}}}}
\newcommand{\ErrorTok}[1]{\textcolor[rgb]{0.64,0.00,0.00}{\textbf{#1}}}
\newcommand{\ExtensionTok}[1]{#1}
\newcommand{\FloatTok}[1]{\textcolor[rgb]{0.00,0.00,0.81}{#1}}
\newcommand{\FunctionTok}[1]{\textcolor[rgb]{0.00,0.00,0.00}{#1}}
\newcommand{\ImportTok}[1]{#1}
\newcommand{\InformationTok}[1]{\textcolor[rgb]{0.56,0.35,0.01}{\textbf{\textit{#1}}}}
\newcommand{\KeywordTok}[1]{\textcolor[rgb]{0.13,0.29,0.53}{\textbf{#1}}}
\newcommand{\NormalTok}[1]{#1}
\newcommand{\OperatorTok}[1]{\textcolor[rgb]{0.81,0.36,0.00}{\textbf{#1}}}
\newcommand{\OtherTok}[1]{\textcolor[rgb]{0.56,0.35,0.01}{#1}}
\newcommand{\PreprocessorTok}[1]{\textcolor[rgb]{0.56,0.35,0.01}{\textit{#1}}}
\newcommand{\RegionMarkerTok}[1]{#1}
\newcommand{\SpecialCharTok}[1]{\textcolor[rgb]{0.00,0.00,0.00}{#1}}
\newcommand{\SpecialStringTok}[1]{\textcolor[rgb]{0.31,0.60,0.02}{#1}}
\newcommand{\StringTok}[1]{\textcolor[rgb]{0.31,0.60,0.02}{#1}}
\newcommand{\VariableTok}[1]{\textcolor[rgb]{0.00,0.00,0.00}{#1}}
\newcommand{\VerbatimStringTok}[1]{\textcolor[rgb]{0.31,0.60,0.02}{#1}}
\newcommand{\WarningTok}[1]{\textcolor[rgb]{0.56,0.35,0.01}{\textbf{\textit{#1}}}}
\usepackage{graphicx}
\makeatletter
\def\maxwidth{\ifdim\Gin@nat@width>\linewidth\linewidth\else\Gin@nat@width\fi}
\def\maxheight{\ifdim\Gin@nat@height>\textheight\textheight\else\Gin@nat@height\fi}
\makeatother
% Scale images if necessary, so that they will not overflow the page
% margins by default, and it is still possible to overwrite the defaults
% using explicit options in \includegraphics[width, height, ...]{}
\setkeys{Gin}{width=\maxwidth,height=\maxheight,keepaspectratio}
% Set default figure placement to htbp
\makeatletter
\def\fps@figure{htbp}
\makeatother
\setlength{\emergencystretch}{3em} % prevent overfull lines
\providecommand{\tightlist}{%
  \setlength{\itemsep}{0pt}\setlength{\parskip}{0pt}}
\setcounter{secnumdepth}{-\maxdimen} % remove section numbering
\ifLuaTeX
  \usepackage{selnolig}  % disable illegal ligatures
\fi

\begin{document}
\maketitle

\hypertarget{problem-1}{%
\section{Problem 1}\label{problem-1}}

For this problem you will need to include some LaTex code. Please
install latex on your computer and then consult Compulsor1.Rmd for hints
how to write formulas in LaTex.

\(E[(y_0 - \hat{f}(x_0))^2]\)
\(=E[y_0^2 -2y_0 \hat{f}(x_0) + \hat{f}(x_0)^2]\)
\(=E[y_0^2] -2E[y_0 \hat{f}(x_0)] + E[\hat{f}(x_0)^2]\)
\(=E[y_0]^2 + Var[y_0] -2E[y_0 \hat{f}(x_0)] + E[\hat{f}(x_0)]^2 + Var[\hat{f}(x_0)]\)
\(=E[f(x_0) + \epsilon]^2 + Var[f(x_0) + \epsilon] -2E[(f(x_0) + \epsilon) \hat{f}(x_0)] + E[\hat{f}(x_0)]^2 + Var[\hat{f}(x_0)]\)
It is assumed that \(\epsilon\) is independent of \(x\), and that
\(E[\epsilon]=0\).
\(=E[f(x_0) + \epsilon]^2 + Var[f(x_0) + \epsilon] -2E[\epsilon \hat{f}(x_0)] - 2E[f(x_0) \hat{f}(x_0)] + E[\hat{f}(x_0)]^2 + Var[\hat{f}(x_0)]\)
\(=E[f(x_0)]^2 + Var[f(x_0)] + Var[\epsilon] -2E[\epsilon \hat{f}(x_0)] - 2E[f(x_0) \hat{f}(x_0)] + E[\hat{f}(x_0)]^2 + Var[\hat{f}(x_0)]\)
\(=E[f(x_0)]^2 - 2E[f(x_0) \hat{f}(x_0)] + E[\hat{f}(x_0)]^2 + Var[\hat{f}(x_0)] + Var[\epsilon]\)
\(=f(x_0)^2 - 2f(x_0) E[\hat{f}(x_0)] + E[\hat{f}(x_0)]^2 + Var[\hat{f}(x_0)] + Var[\epsilon]\)
\(=(f(x_0) -E[\hat{f}(x_0)])^2 + Var[\hat{f}(x_0)] + Var[\epsilon]\)
Where \(Var[\epsilon]\) is the irreducible error, \(Var[\hat{f}(x_0)]\)
is the variance of the prediction, and \((f(x_0) -E[\hat{f}(x_0)])^2\)

\hypertarget{a}{%
\subsection{a)}\label{a}}

\hypertarget{b}{%
\subsection{b)}\label{b}}

\hypertarget{c}{%
\subsection{c)}\label{c}}

\hypertarget{d}{%
\subsection{d)}\label{d}}

\hypertarget{e}{%
\subsection{e)}\label{e}}

\hypertarget{problem-2}{%
\section{Problem 2}\label{problem-2}}

Here is a code chunk:

\begin{Shaded}
\begin{Highlighting}[]
\FunctionTok{library}\NormalTok{(palmerpenguins) }\CommentTok{\# Contains the data set "penguins".}
\FunctionTok{data}\NormalTok{(penguins)}
\FunctionTok{head}\NormalTok{(penguins)}
\end{Highlighting}
\end{Shaded}

\begin{verbatim}
## # A tibble: 6 x 8
##   species island bill_length_mm bill_depth_mm flipper_length_~ body_mass_g sex  
##   <fct>   <fct>           <dbl>         <dbl>            <int>       <int> <fct>
## 1 Adelie  Torge~           39.1          18.7              181        3750 male 
## 2 Adelie  Torge~           39.5          17.4              186        3800 fema~
## 3 Adelie  Torge~           40.3          18                195        3250 fema~
## 4 Adelie  Torge~           NA            NA                 NA          NA <NA> 
## 5 Adelie  Torge~           36.7          19.3              193        3450 fema~
## 6 Adelie  Torge~           39.3          20.6              190        3650 male 
## # ... with 1 more variable: year <int>
\end{verbatim}

\hypertarget{a-1}{%
\subsection{a)}\label{a-1}}

\hypertarget{b-1}{%
\subsection{b)}\label{b-1}}

\hypertarget{c-1}{%
\subsection{c)}\label{c-1}}

\hypertarget{problem-3}{%
\section{Problem 3}\label{problem-3}}

\hypertarget{problem-4}{%
\section{Problem 4}\label{problem-4}}

\end{document}
